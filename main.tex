\documentclass[conference]{IEEEtran}
\IEEEoverridecommandlockouts
% The preceding line is only needed to identify funding in the first footnote. If that is unneeded, please comment it out.
%\usepackage{cite}
\usepackage{amsmath,amssymb,amsfonts}
\usepackage{algorithmic}
\usepackage{graphicx}
\usepackage{textcomp}
\usepackage{xcolor}
\usepackage[backref=true,defernumbers=true,citestyle=numeric]{biblatex}
\usepackage{fancyhdr}
% \usepackage{listings}
% \usepackage{hyperref}

% \definecolor{vgreen}{RGB}{104,180,104}
%     \definecolor{vblue}{RGB}{49,49,255}
%     \definecolor{vorange}{RGB}{255,143,102}
%     \lstdefinestyle{verilog-style}
%     {
%         language=Verilog,
%         basicstyle=\small,
%         keywordstyle=\color{vblue},
%         identifierstyle=\color{black},
%         commentstyle=\color{vgreen},
%         numbers=left,
%         numberstyle={\tiny \color{black}},
%         numbersep=10pt,
%         tabsize=8
%     }


% 设置页脚样式
\pagestyle{fancy}
% 清空页眉和页脚的默认设置
\fancyhf{}
% 在页脚中间插入页码
\fancyfoot[C]{\thepage}

% 去掉页眉和页脚的横线
\renewcommand{\headrulewidth}{0pt}
\renewcommand{\footrulewidth}{0pt}

\addbibresource{references.bib}
% \def\BibTeX{{\rm B\kern-.05em{\sc i\kern-.025em b}\kern-.08em
%     T\kern-.1667em\lower.7ex\hbox{E}\kern-.125emX}}
\begin{document}

\title{Wishbone Bus Implementation\\
{\footnotesize \textsuperscript{}ECE4810J 2024Fall System-on-Chip Design Final Project}
}

\author{
\IEEEauthorblockN{Lanqi Ding}
\IEEEauthorblockA{\textit{Shanghai Jiao Tong University} \\
\textit{UM-SJTU Joint Institute}\\
dinglanqi20021215@sjtu.edu.cn}
\and
\IEEEauthorblockN{Zhangsiyuan Jin}
\IEEEauthorblockA{\textit{Shanghai Jiao Tong University} \\
\textit{UM-SJTU Joint Institute}\\
jzsiyuan@sjtu.edu.cn}
\and
\IEEEauthorblockN{Muzi Li}
\IEEEauthorblockA{\textit{Shanghai Jiao Tong University} \\
\textit{UM-SJTU Joint Institute}\\
muzili529@sjtu.edu.cn}

}

\maketitle

\begin{abstract}
This document presents the implementation of the Wishbone bus architecture using High-Level Synthesis (HLS). The Wishbone bus protocol, known for its simplicity and flexibility, is widely adopted in System-on-Chip (SoC) designs to enhance the efficiency of data exchange between different Intellectual Property (IP) cores. Our project involved three major versions of the Wishbone bus implementation, each adding new features and improving performance. Version 1 (v1) focused on primitive read/write operations, Version 2 (v2) introduced support for \texttt{BASEADDR}, \texttt{SIZE}, and \texttt{OFFSET}, and Version 3 (v3) added the \texttt{sel} signal for partial byte operations. The implementation was thoroughly tested using Vitis HLS, and performance metrics such as speed, latency, and throughput were analyzed. The introduction of the \texttt{sel} signal in v3 resulted in a slight increase in timing from 12ns to 14ns, highlighting the trade-off between functionality and performance. Future work includes exploring the use of a custom signal line, \texttt{TAGN}, for enhanced debugging, QoS management, security, and protocol extensions. Overall, this project demonstrates the effectiveness of using HLS for designing complex on-chip interconnect architectures, providing a solid foundation for future developments in this area.

The complete source code for this project is available at \url{https://github.com/LEXIEDING/ECE4810J_2024FALL/tree/feat/wishbone}. This repository includes the assignment specification, design ideas, detailed information about the WISHBONE interface and read/write operation logic, HLS design code, version information (including Vitis Reports, Vivado diagrams, and Vitis Application results), as well as a simple installation guide.

\end{abstract}

\begin{IEEEkeywords}
Wishbone bus, High-Level Synthesis, System-on-Chip, Vitis HLS, partial byte operations, \texttt{sel} signal, \texttt{TAGN} signal, performance optimization
\end{IEEEkeywords}

\section{Introduction}
In the field of digital system design, particularly in the design of System-on-Chips (SoCs), the efficiency and reliability of communication between different hardware components are crucial. The Wishbone bus protocol, as a lightweight, open-source on-chip interconnect architecture, is widely adopted due to its simplicity and flexibility. Our project aims to implement this protocol to enhance the efficiency of data exchange between different Intellectual Property (IP) cores in SoCs, ensure the stability of communication, and meet the demands of modern electronic systems for high performance and scalability. The Wishbone bus, through its master-slave communication model, allows the master device to control data transfer while the slave device responds to requests, a model that excels in ensuring data consistency and reducing bus conflicts. Our solution will involve the detailed design and implementation of the Wishbone bus interface, including the maintenance of signal integrity, support for diverse data transfer methods, performance optimization, and considerations for compatibility and scalability. Our goal is to provide an efficient and reliable implementation of the Wishbone bus that not only meets current technical standards but can also adapt to future technological developments. Through this implementation, we expect to enhance system performance while simplifying the SoC design process, offering an effective solution for digital system design.\cite{Sharma_2012}



\begin{figure}[h]
    \centering
    \includegraphics[width=0.3\textwidth]{wishbone.png}
    \caption{Wishbone Bus}
    \label{wishbone}
\end{figure}




\section{Related Work}
Our project on the Wishbone bus architecture is situated within a rich landscape of existing research and development in the field of on-chip interconnection architectures. Prior work has explored various aspects of bus architectures, including their performance implications for System-on-Chip (SoC) designs. Notably, Sharma and Kumar's survey, "Wishbone bus Architecture - A Survey and Comparison," provides a comprehensive overview of the Wishbone bus architecture and its comparison with other on-chip bus architectures such as AMBA, CoreConnect, and Avalon. Their study highlights the Wishbone bus's advantages, such as its flexible arbitration scheme and additional data transfer cycle (Read-Modify-Write cycle)\cite{Sharma_2012}, which aligns with our focus on enhancing the efficiency of data transfer within SoCs.

In addition to Sharma and Kumar's work, the OpenCores community has contributed several projects related to Wishbone, including an AXI to Wishbone bridge, which demonstrates the practical application and adaptability of the Wishbone protocol in connecting different bus systems\cite{opencores_wishbone}. These projects showcase the versatility of Wishbone in integrating with other protocols, which is relevant to our work as we aim to leverage Wishbone's compatibility features.

Furthermore, the "Double Wishbone Suspension Design Based on Design for Six Sigma (DFSS)" research, while not directly related to bus architectures, employs a robust design methodology that could be analogously applied to our project for optimizing the performance and reliability of the Wishbone bus implementation\cite{10.1145/3421766.3421866}. This study's approach to minimizing variations and enhancing robustness provides a methodological perspective that may inform our optimization strategies.


While our work is not entirely novel, it contributes to the existing body of knowledge by focusing specifically on the Wishbone bus and its integration within the context of modern SoC design challenges. We build upon these foundational works to address specific performance and optimization objectives that are critical for the advancement of on-chip interconnection technologies.



\section{Methods}
In this project, we worked to develop a complete Wishbone bus system by the high level synthesis (HLS) design methods. HLS makes the design faster, allowing us to iterate and evaluate products in a shorter amount of time. Our design process follows the guidelines in the Lab1 manual\cite{ECE4810J_FA2024_Lab1} to build a Wishbone IP block.

\subsection{Overall Design Approach}

In our project, we utilized High-Level Synthesis (HLS) to implement the Wishbone bus architecture, opting for a streamlined design and verification process. HLS, with its capability to describe hardware behavior using high-level languages like C and C++, significantly reduced our development time and complexity compared to traditional RTL design. By employing HLS for rapid prototyping, we efficiently explored multiple design options and established behavior-level models of the Wishbone bus. This method not only accelerated the design cycle but also ensured a comprehensive understanding and verification of the design before delving into RTL complexities.

Our choice of HLS underscored a commitment to innovation and efficiency, aligning with the industry's trend towards more accessible and faster hardware development methodologies. This approach, while concise, encapsulates the essence of our project's contribution to the field of SoC design.


\subsection{Interface Design}
The Wishbone bus is a flexible and efficient on-chip interconnect (SoC) protocol, achieving communication between master and slave devices through a series of meticulously designed interfaces as shown in Figure \ref{interface}. Here is a detailed description of the main interfaces of the Wishbone bus and their purposes:

\begin{figure}[h]
    \centering
    \includegraphics[width=0.4\textwidth]{cover.png}
    \caption{Interface Details of Wishbone}
    \label{interface}
\end{figure}

\begin{itemize}
    \item \textbf{CLK\_I} (Clock Signal): Provides a synchronous clock signal for all devices on the Wishbone bus, ensuring the synchronization of data transmission.
    
    \item \textbf{RST\_I} (Reset Signal): Used to asynchronously reset all devices on the Wishbone bus, typically triggered by an external control signal.
    
    \item \textbf{ADR\_O/ADR\_I} (Address Lines): Used to transmit address information from the master to the slave device, with ADR\_O indicating the address signal output by the master and ADR\_I representing the address signal input to the slave.
    
    \item \textbf{DAT\_O/DAT\_I} (Data Lines): For transmitting data between the master and slave devices, where DAT\_O signifies the data output by the master, and DAT\_I represents the data input from the slave.
    
    \item \textbf{WE\_O/WE\_I} (Write Enable Signals): Control the direction of data transfer, with WE\_O output by the master to indicate a write operation (1 for write, 0 for read), and WE\_I is an input to the slave, indicating whether the slave accepts the write operation.
    
    \item \textbf{SEL\_O/SEL\_I} (Data Line Select Signals): Used for byte selection, selecting which part of the data lines is active on a byte-by-byte basis. SEL\_O is output by the master, and SEL\_I is an input to the slave.
    
    \item \textbf{STB\_O/STB\_I} (Strobe Signals): Indicate the beginning of a bus operation. STB\_O is output by the master, and STB\_I is an input to the slave.
    
    \item \textbf{CYC\_O/CYC\_I} (Cycle Signals): Indicate the start and end of a bus cycle. CYC\_O is output by the master, signaling the start of a bus cycle, and CYC\_I is an input to the slave, signaling the end of the cycle.
    
    \item \textbf{ACK\_O/ACK\_I} (Acknowledge Signals): Used to indicate the successful completion of a bus operation. ACK\_O is output by the slave, and ACK\_I is an input to the master.
    
    \item \textbf{TAGN\_O/TAGN\_I} (Tag Signals): For conveying user-defined information, which can be utilized to implement more complex communication protocols.
\end{itemize}

Additionally, the Wishbone bus supports various connection methods, including point-to-point, data flow, shared bus, and crossbar connections, to accommodate different system architecture needs. For instance, point-to-point connections are suitable for single master-slave communications, while shared bus and crossbar connections allow multiple masters to communicate with one or multiple slaves.

\subsection{Data Transmission}
\subsubsection{Read Operation}
The read operation is initiated by the master device. It sets the address signal ADR\_O (address output), the data select signal SEL\_O (byte select), and the strobe signal STB\_O to commence a read operation, while grounding the write enable signal WE\_O to explicitly indicate that this is a read operation. At the rising edge of the clock cycle 0, the master asserts the strobe signal STB\_O and the cycle signal CYC\_O to initiate a transaction, marking the beginning of the operation.

Before the subsequent clock cycle's rising edge arrives, the slave device detects the master's initiation signal and responds by driving the requested data onto the data line DAT\_O and asserting the acknowledge signal ACK\_O in response to the master's request. At the rising edge of the clock cycle 1, once the master detects the ACK\_O signal to be high, it samples the data on the data line DAT\_I, and then concludes the operation by de-asserting CYC\_O and STB\_O, signifying the end of a complete read cycle.

This process ensures that the master device can read data from the slave device. After the slave confirms that the data is ready, it informs the master that it can proceed with data sampling via the ACK\_O signal. Once the master device has received and sampled the data, it halts the current transaction, preparing for the next operation or waiting for new instructions. This coordinated communication mechanism is at the core of the Wishbone bus protocol, allowing for efficient data exchange between different components in complex SoC designs.

\subsubsection{Write Operation}
The write operation is meticulously orchestrated by the master device. At the rising edge of the clock cycle 0, the master initiates the transaction by setting the address signal ADR\_O (address output), data signal DAT\_O (data output), data select signal SEL\_O (byte select), strobe signal STB\_O, and the write enable signal WE\_O, which is asserted to indicate a write operation. This configuration signals the commencement of the transaction.

Upon recognizing the master's command, the slave device promptly latches the data from DAT\_I with the provided data and prepares to acknowledge the transaction. Prior to the subsequent rising edge of the clock, the slave device, having recognized the master's command, locks in the data onto the DAT\_O line and asserts the acknowledge signal ACK\_O to confirm receipt of the data and the successful initiation of the write operation.

At the rising edge of the clock cycle 1, the master, upon detecting the assertion of ACK\_O, proceeds to sample the acknowledgment signal ACK\_I. With the confirmation of the successful write operation, the master then deactivates the cycle signal CYC\_O and the strobe signal STB\_O, effectively finalizing the write operation. This coordinated exchange ensures that the data is accurately transmitted from the master to the slave, and the slave's acknowledgment signifies the completion of the data transfer.

This methodical and precise sequence of signals and responses guarantees the integrity and reliability of data transmission within the Wishbone bus system. The write operation is a fundamental aspect of the communication protocol, enabling the master device to update or store data in the slave device's memory or registers.


\subsection{High-Level Synthesis (HLS) Implementation}
The core functionality of the Wishbone bus was implemented using C++ in the HLS environment. The main components included:
\begin{itemize}
    \item \textbf{WishboneMaster}: Handles the generation of bus cycles and data transfer requests.
    \item \textbf{WishboneSlave}: Responds to the master's requests and performs read/write operations.
    \item \textbf{WishboneArbiter}: Manages the arbitration between multiple slaves and ensures correct data routing.
\end{itemize}


We divided our implementation into three main versions, each building upon the previous one to add more features and improve performance:

\begin{itemize}
    \item \textbf{Version 1 (v1)}: Primitive Read/Write Wishbone Master/Slave Implementation.
    \item \textbf{Version 2 (v2)}: Added support for \texttt{BASEADDR}, \texttt{SIZE}, and \texttt{OFFSET}.
    \item \textbf{Version 3 (v3)}: Introduced the \texttt{sel} signal for partial byte operations.
\end{itemize}


Each version was thoroughly tested using corresponding testbenches to ensure correctness and performance. The HLS tool used for this project was Vitis HLS, which provided an efficient workflow for synthesizing and verifying our designs.


\begin{figure}[htbp]
    \centering
    \includegraphics[width=0.4\textwidth]{v3_vitis_hls_interface_definition.png}
    \caption{Vitis HLS Interface Definition for v3}
    \label{v3_vitis_hls_interface_definition}
\end{figure}

\begin{figure}[htbp]
    \centering
    \includegraphics[width=0.4\textwidth]{v3_block_design.png}
    \caption{Block Design for v3}
    \label{fig:v3_block_design}
\end{figure}

\begin{figure}[htbp]
    \centering
    \includegraphics[width=0.4\textwidth]{v3_implemented_design_highlight.png}
    \caption{Implemented Design Highlight for v3}
    \label{fig:v3_implemented_design_highlight}
\end{figure}

% \subsection{Register Transfer Level (RTL) Implementation}
% In the RTL implementation section of our paper, we provide an overview of the hardware description language implementation of the Wishbone interface, designed to facilitate efficient communication between master and slave devices within a System-on-Chip (SoC) environment. The RTL module, `wishbone\_interface`, implements the Wishbone protocol through a series of ports that include clock, reset, cycle, strobe, write enable, address, data input, and output signals, as well as acknowledge and error signals.

% The write operation is initiated by the master device at the rising edge of the clock by setting the address, data, and control signals. Upon detecting the command, the slave device responds by latching the data and asserting an acknowledge. The read operation follows a similar pattern, where the master device sets the address and control signals, and the slave provides the data and an acknowledgment. The error signal is utilized to indicate any anomalies occurring during the data transfer process. The corresponding code is shown in Figure \ref{read_write_code}.

% \begin{figure}[htbp]
%     \centering
%     \includegraphics[width=0.4\textwidth]{read_write_code.png}
%     \caption{RTL Code of Read and Write}
%     \label{read_write_code}
% \end{figure}



\section{Methodology}

The evaluation methodology of the implemented Wishbone protocol interface is based on a combination of functional testing, coverage analysis, and performance profiling. These steps were designed to ensure that the design is robust, meets functional requirements, and operates efficiently under realistic scenarios.

\subsection{Functional Testing}
Functional testing was performed to validate the correctness of the design. Test cases covered:
\begin{itemize}
    \item Basic read and write operations on the Wishbone master and slave interfaces, verifying correct address and data handling.
    \item Address decoding logic in the arbiter, ensuring proper routing of transactions to the corresponding slaves based on address ranges.
    \item Handling of byte select signals (\texttt{sel}) to verify partial data manipulation within the memory.
    \item Response timing with \texttt{ack} signal assertion to validate synchronous behavior.
\end{itemize}
The tests simulated various realistic usage scenarios and edge cases to ensure the design operates correctly under all conditions.

\subsection{Coverage Analysis}
Code coverage metrics were employed to ensure all logical paths in the design were tested. Special attention was given to:
\begin{itemize}
    \item Address boundaries, ensuring transactions targeting out-of-range addresses were correctly handled.
    \item Read and write operation sequences, confirming consistent data integrity and synchronization.
    \item Concurrent access scenarios to evaluate arbitration correctness and priority handling.
\end{itemize}

\subsection{Performance Profiling}
The performance of the design was profiled to evaluate throughput and latency under varying workloads. By using hardware synthesis tools (e.g., Vivado HLS), latency for data transfers and memory operations was measured. Resource utilization and timing reports were reviewed to identify potential bottlenecks or areas for optimization.

\subsection{Fairness and Validity}

The evaluation methodology aligns closely with the design and implementation of the Wishbone protocol interface to ensure a comprehensive and fair assessment. Specific measures include:

\begin{itemize}
    \item \textbf{Realistic Testing Scenarios:} The evaluation utilized practical tests covering various address ranges, including valid and invalid memory addresses, as well as boundary conditions. For instance, the code tests write and read operations across slave boundaries (e.g., addresses around 0x00000400) to verify proper arbitration and data integrity.
    
    \item \textbf{Full Byte-Select Verification:} The tests included partial byte operations using the \texttt{sel} signal to ensure correct handling of individual bytes. For example, operations like writing to the least significant byte (\texttt{sel = 0x1}) and reading back all bytes were validated to confirm accuracy.

    \item \textbf{Validation Against Protocol Specifications:} The \texttt{ack} signal behavior was explicitly tested to confirm proper acknowledgment of valid transactions and rejection of invalid or out-of-range accesses (e.g., addresses beyond 0x00000800).

    \item \textbf{Edge Case and Error Handling:} The design was subjected to edge cases such as unaligned addresses and simultaneous operations on different slaves to verify robust error handling and arbitration. Invalid conditions were correctly flagged without unintended side effects.

    \item \textbf{Platform-Independent Checks:} The implementation leverages modular functions such as \texttt{write\_data()} and \texttt{read\_data()} to standardize operations, ensuring consistency across various scenarios and hardware configurations.

\end{itemize}

By tailoring the tests to match the specific implementation details, this methodology ensures a valid and unbiased evaluation. The approach systematically covers all potential operational conditions, enhancing confidence in the design's reliability and compliance with the Wishbone protocol.






\section{Evaluation}

The evaluation of our Wishbone bus implementation involved several key aspects, including functional verification, performance analysis, and a detailed examination of the timing and resource utilization.

\subsection{Functional Verification}
Functional verification was performed using the testbenches developed for each version of the implementation. The testbenches covered a wide range of scenarios, including basic read/write operations, partial byte operations using the \texttt{sel} signal, boundary condition testing, cross-slave boundary operations, and handling of invalid addresses. The test results, as shown in the Vitis Serial Terminal output, demonstrate successful operations across various memory addresses and \texttt{sel} signal configurations.

\begin{figure}[h]
    \centering
    \includegraphics[width=0.4\textwidth]{v3_vitis_application_outcome.png}
    \caption{Vitis HLS Application Outcome for v3}
    \label{fig:v3_vitis_application_outcome}
\end{figure}

\subsection{Performance Analysis}
Performance metrics such as speed, latency, and throughput were measured to evaluate the efficiency of the Wishbone bus implementation. Each version of the implementation was synthesized using Vitis HLS, and the Timing Estimate from the Synthesis Summary was analyzed.

\subsubsection{Timing Estimate}
The Co-simulation Report for version v3 shows a Target Timing Estimate of 14ns, with an estimated uncertainty of 9.297ns to 3.78ns. This indicates a robust performance with a high degree of confidence in the timing estimates. The Synthesis Summary Report further details the performance with an Average Latency of 79ns, a Minimum Latency of 4ns, and a Maximum Latency of 87ns for the top module.

\begin{figure}[htbp]
    \centering
    \includegraphics[width=0.4\textwidth]{v3_vitis_hls_co-simulation_report.png}
    \caption{Vitis HLS Co-simulation Report for v3}
    \label{fig:v3_vitis_hls_co-simulation_report}
\end{figure}

\begin{figure}[h]
    \centering
    \includegraphics[width=0.4\textwidth]{v3_vitis_hls_solution_report.png}
    \caption{Vitis HLS Solution Report for v3}
    \label{fig:v3_vitis_hls_solution_report}
\end{figure}

\subsubsection{Resource Utilization}
The resource utilization was also analyzed, with the Synthesis Summary Report indicating the use of 375 LUTs and 694 flip-flops for the top module. The process module utilized 94 LUTs and 214 flip-flops. These resource estimates are crucial for understanding the implementation's efficiency and its potential impact on the overall system design. And as shown in Figure \ref{v1_ComparedWith_v3},the increase in FFs and LUTs for v3 compared to v1 is a direct result of the additional logic required to support the enhanced features. More complex control mechanisms and data handling lead to higher resource usage.
Both solutions use the same amount of BRAM, suggesting that the memory requirements have not significantly changed with the added features.


\subsubsection{Trade-off Analysis}
The introduction of the \texttt{sel} signal in version v3, while providing significant benefits in terms of flexibility and functionality, resulted in a slight increase in timing from 12ns to 14ns as shown in Figure \ref{v1_ComparedWith_v3}. This increase can be attributed to the additional logic required to handle partial byte operations, which introduces more complexity and potential delays in the data path.

\begin{figure}[h]
    \centering
    \includegraphics[width=0.4\textwidth]{v1_ComparedWith_v3.png}
    \caption{Vitis HLS Comparison Report for v1 and v3}
    \label{v1_ComparedWith_v3}
\end{figure}

\subsubsection{Latency}
v1 has a minimum latency of 2 cycles, which is higher than v3's 1 cycle, indicating that v3 can initiate a new transaction more quickly in the best-case scenario. And the maximum latency for v3 is higher than v1, which is expected due to the additional logic required to handle the sel signal and address decoding based on BASEADDR, SIZE, and OFFSET.


The sel signal can be used in many aspects such as Memory-Mapped I/O, Image Processing, Audio Processing and Embedded System as shown in Figure \ref{sel_application}. \cite{mmio}

\begin{figure}[h] 
    \centering
    \includegraphics[width=0.4\textwidth]{sel_application-1.png}
    \includegraphics[width=0.4\textwidth]{sel_application-2.png}
    \caption{Practical Applications of the sel Signal}
    \label{sel_application}
\end{figure}

\textbf{When to Use \texttt{sel} Signal:}
The \texttt{sel} signal is particularly useful in scenarios where:
\begin{itemize}
    \item \textbf{Partial Byte Operations}: Applications that require frequent partial byte read/write operations can benefit from the \texttt{sel} signal, as it allows for more efficient data manipulation without the need for additional software handling.
    \item \textbf{Memory-Mapped I/O}: In systems with memory-mapped I/O, the \texttt{sel} signal can be used to access specific bytes within a word, improving the granularity of control and data access.
    \item \textbf{Performance Optimization}: For applications where the overhead of additional timing is acceptable, the \texttt{sel} signal can provide enhanced functionality and flexibility, leading to overall system performance improvements.
\end{itemize}

Based on the trade-off, we build a simple model to compare the total operation time with and without the sel signal for different read/write ratios. As shown in Figure \ref{sel}, the time for both read and write operations without the 'sel' signal is assumed to be 36 ns each, calculated as 12 ns multiplied by 3, and the overall equivalent time for an operation without 'sel' includes this duration plus an additional 12 ns dedicated to masking operations. The time for a read operation with the 'sel' signal is calculated as 28 ns, which is 14 ns multiplied by 2, while the time for a write operation with the 'sel' signal is 84 ns, calculated as 14 ns multiplied by 6.


\begin{figure}[h] 
    \centering
    \includegraphics[width=0.4\textwidth]{sel.png}
    \caption{Comparison Time within and without sel for Different Read/Write Ratios}
    \label{sel}
\end{figure}



In conclusion, the Wishbone bus implementation demonstrates a balance between functionality and performance. The decision to use the \texttt{sel} signal should be based on a careful analysis of the specific requirements and constraints of the target application, considering the trade-offs in timing and resource utilization.



\section{Optimization}
In the optimization segment of our project, we focused on enhancing the performance of the Wishbone bus interface through the implementation of burst operations. The additional signals CTI\_I/O (Cycle Type Identifier) and BTE\_I/O (Burst Type Extension) play a pivotal role in this process. The CTI\_I/O signal is utilized to denote the type of cycle, with specific values, such as '111', indicating the End-of-Burst. Meanwhile, BTE\_I/O is employed to define the burst pattern, whether it's linear or wrap. The burst read and write transactions are shown in Figure \ref{Burst Read Transaction} and \ref{Burst Write Transaction} \cite{csdn84531533}.

\begin{figure}[h]
    \centering
    \includegraphics[width=0.4\textwidth]{burst-1.png}
    \caption{Burst Read Transaction}
    \label{Burst Read Transaction}
\end{figure}

\begin{figure}[h]
    \centering
    \includegraphics[width=0.4\textwidth]{burst-2.png}
    \caption{Burst Write Transaction}
    \label{Burst Write Transaction}
\end{figure}

The implementation steps initiated with the assertion of STB\_O and CYC\_O to start a burst, followed by the negation of WE\_O for read operations or its assertion for write operations. The CTI\_I/O signal is strategically used to signal the cycle type, while BTE\_I/O determines the burst pattern, contributing to the overall efficiency of the data transfer.

The incorporation of burst operations has yielded several benefits, including a significant performance boost due to the reduced number of transactions required for data transfer. It has also led to a reduction in latency, as burst operations minimize the time spent on initiating and concluding data transfers. Furthermore, the efficiency gain is evident in the optimized use of bus bandwidth and reduced idle times. The flexibility of burst operations allows for more complex and variable data access patterns, supporting a wider range of use cases and enhancing the overall complexity of operations that can be managed by the Wishbone bus.

This optimization strategy has been a cornerstone in our project, providing a robust solution that not only meets but exceeds the demands of modern SoC designs where data throughput and system efficiency are paramount.

What's more, TAGN\_I/O (Tag Signals) is not included in our current implementation, but we can enrich our feature set by enabling it for several advanced functionalities.

Implementing TAGN\_I/O introduces a sophisticated error detection and handling mechanism to our system, enabling the tagging and routing of error messages back to their source. This enhancement allows for quicker identification and resolution of issues, thereby improving system reliability. Additionally, TAGN\_I/O facilitates dynamic priority management, which is crucial in multi-tasking environments. By assigning different priorities to various operations, we can ensure that certain tasks are executed in a timely manner, enhancing overall system performance. Furthermore, TAGN\_I/O enables differential data identification, distinguishing between command, status, and payload data transfers. This capability allows the system to process each type of data more efficiently, as it can handle them appropriately according to their nature. In summary, TAGN\_I/O's implementation enriches our system's capabilities by enhancing error handling, managing dynamic priorities, and identifying different data types, leading to improved data processing efficiency and system responsiveness.




\section{Statement of Work}
For the successful completion of the Wishbone Bus Implementation project, our team of three members has divided the work into three parts: System Design and Integration, Hardware Implementation and Debugging, Performance Analysis and Optimization.

\begin{itemize}
    \item Zhangsiyuan Jin: System Design and Integration

    She conducted a preliminary research on the Wishbone bus architecture through data collection and literature search, focusing on its protocols and interfaces. Then she designed a Wishbone compliant system architecture and integrated the bus with other SoC components. In addition, she developed a testbench and conducted simulations to verify the performance of the bus.

    \item Lanqi Ding: Hardware Implementation and Debugging

    She was responsible for crafting the hardware description language (HDL) code for the Wishbone bus interface, bringing the design to life. She expertly implemented the arbitration logic and data transfer mechanisms that are essential for the bus's operation, ensuring efficient communication between components. When challenges arose during simulations and testing, Lanqi diligently debugged the design, pinpointing and resolving issues to maintain project momentum. 

    \item Muzi Li: Performance Analysis and Optimization

    She played a critical role in assessing the performance of the Wishbone bus implementation, diligently measuring key metrics like speed, latency, and throughput to ensure the system's efficiency. She identified performance bottlenecks and proposed targeted optimization strategies to boost the bus's capabilities. Working closely with Member A and Member B, they collaboratively implemented these enhancements, leading to significant improvements in the system's overall performance. Additionally, Member C was tasked with compiling the final report, which comprehensively detailed the performance analysis findings and the tangible benefits achieved through the implemented optimizations.

\end{itemize}


\section{Conclusion}
This project successfully implemented the Wishbone bus architecture and improved design efficiency and performance through advanced integration (HLS) technology. Our work not only demonstrates the effectiveness of designing complex on-chip interconnect architectures using HLS, but also provides a solid foundation for future developments in this area. With three major versions of the Wishbone bus implementation, we have progressively added new features and improved performance. v1 focused on basic read and write operations, v2 introduced support for BASEADDR, SIZE, and OFFSET, and v3 added sel signals to support partial byte operations. Implementation tests and performance analysis with Vitis HLS show that the introduction of sel signals, while increasing the timing from 12ns to 14ns, is a necessary trade-off between functionality and performance. Our work not only implements the Wishbone bus, but also explores future extensions such as TAGN signaling and burst operations to enhance system performance and reliability. Overall, our project strikes a balance between functionality and performance, providing a comprehensive evaluation and effective solution for the Wishbone bus protocol interface in SoC design.









\printbibliography[title={References},resetnumbers=true,heading=subbibliography]
\clearpage
\end{document}
