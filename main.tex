\documentclass[conference]{IEEEtran}
\IEEEoverridecommandlockouts
% The preceding line is only needed to identify funding in the first footnote. If that is unneeded, please comment it out.
%\usepackage{cite}
\usepackage{amsmath,amssymb,amsfonts}
\usepackage{algorithmic}
\usepackage{graphicx}
\usepackage{textcomp}
\usepackage{xcolor}
\usepackage[backref=true,defernumbers=true,citestyle=numeric]{biblatex}
\usepackage{fancyhdr}
% \usepackage{listings}
% \usepackage{hyperref}

% \definecolor{vgreen}{RGB}{104,180,104}
%     \definecolor{vblue}{RGB}{49,49,255}
%     \definecolor{vorange}{RGB}{255,143,102}
%     \lstdefinestyle{verilog-style}
%     {
%         language=Verilog,
%         basicstyle=\small,
%         keywordstyle=\color{vblue},
%         identifierstyle=\color{black},
%         commentstyle=\color{vgreen},
%         numbers=left,
%         numberstyle={\tiny \color{black}},
%         numbersep=10pt,
%         tabsize=8
%     }


% 设置页脚样式
\pagestyle{fancy}
% 清空页眉和页脚的默认设置
\fancyhf{}
% 在页脚中间插入页码
\fancyfoot[C]{\thepage}

% 去掉页眉和页脚的横线
\renewcommand{\headrulewidth}{0pt}
\renewcommand{\footrulewidth}{0pt}

\addbibresource{references.bib}
% \def\BibTeX{{\rm B\kern-.05em{\sc i\kern-.025em b}\kern-.08em
%     T\kern-.1667em\lower.7ex\hbox{E}\kern-.125emX}}
\begin{document}

\title{Wishbone Bus Implementation\\
{\footnotesize \textsuperscript{}ECE4810J 2024Fall System-on-Chip Design Final Project}
}

\author{
\IEEEauthorblockN{Lanqi Ding}
\IEEEauthorblockA{\textit{ShanghaiJiaoTong University} \\
\textit{UM-Joint Institute}\\
dinglanqi20021215@sjtu.edu.cn}
\and
\IEEEauthorblockN{Zhangsiyuan Jin}
\IEEEauthorblockA{\textit{ShanghaiJiaoTong University} \\
\textit{UM-Joint Institute}\\
jzsiyuan@sjtu.edu.cn}
\and
\IEEEauthorblockN{Muzi Li}
\IEEEauthorblockA{\textit{ShanghaiJiaoTong University} \\
\textit{UM-Joint Institute}\\
muzili529@sjtu.edu.cn}

}

\maketitle

\begin{abstract}
This document 
\end{abstract}

\begin{IEEEkeywords}
Wishbone, 
\end{IEEEkeywords}

\section{Introduction}
In the field of digital system design, particularly in the design of System-on-Chips (SoCs), the efficiency and reliability of communication between different hardware components are crucial. The Wishbone bus protocol, as a lightweight, open-source on-chip interconnect architecture, is widely adopted due to its simplicity and flexibility. Our project aims to implement this protocol to enhance the efficiency of data exchange between different Intellectual Property (IP) cores in SoCs, ensure the stability of communication, and meet the demands of modern electronic systems for high performance and scalability. The Wishbone bus, through its master-slave communication model, allows the master device to control data transfer while the slave device responds to requests, a model that excels in ensuring data consistency and reducing bus conflicts. Our solution will involve the detailed design and implementation of the Wishbone bus interface, including the maintenance of signal integrity, support for diverse data transfer methods, performance optimization, and considerations for compatibility and scalability. Our goal is to provide an efficient and reliable implementation of the Wishbone bus that not only meets current technical standards but can also adapt to future technological developments. Through this implementation, we expect to enhance system performance while simplifying the SoC design process, offering an effective solution for digital system design.\cite{Sharma_2012}



\begin{figure}[h]
    \centering
    \includegraphics[width=0.3\textwidth]{wishbone.png}
    \caption{Wishbone Bus}
    \label{wishbone}
\end{figure}




\section{Related Work}
Our project on the Wishbone bus architecture is situated within a rich landscape of existing research and development in the field of on-chip interconnection architectures. Prior work has explored various aspects of bus architectures, including their performance implications for System-on-Chip (SoC) designs. Notably, Sharma and Kumar's survey, "Wishbone bus Architecture - A Survey and Comparison," provides a comprehensive overview of the Wishbone bus architecture and its comparison with other on-chip bus architectures such as AMBA, CoreConnect, and Avalon. Their study highlights the Wishbone bus's advantages, such as its flexible arbitration scheme and additional data transfer cycle (Read-Modify-Write cycle)\cite{Sharma_2012}, which aligns with our focus on enhancing the efficiency of data transfer within SoCs.

In addition to Sharma and Kumar's work, the OpenCores community has contributed several projects related to Wishbone, including an AXI to Wishbone bridge, which demonstrates the practical application and adaptability of the Wishbone protocol in connecting different bus systems\cite{opencores_wishbone}. These projects showcase the versatility of Wishbone in integrating with other protocols, which is relevant to our work as we aim to leverage Wishbone's compatibility features.

Furthermore, the "Double Wishbone Suspension Design Based on Design for Six Sigma (DFSS)" research, while not directly related to bus architectures, employs a robust design methodology that could be analogously applied to our project for optimizing the performance and reliability of the Wishbone bus implementation\cite{10.1145/3421766.3421866}. This study's approach to minimizing variations and enhancing robustness provides a methodological perspective that may inform our optimization strategies.


While our work is not entirely novel, it contributes to the existing body of knowledge by focusing specifically on the Wishbone bus and its integration within the context of modern SoC design challenges. We build upon these foundational works to address specific performance and optimization objectives that are critical for the advancement of on-chip interconnection technologies.



\section{Methods}
In this project, we worked to develop a complete Wishbone bus system by integrating advanced synthesis (HLS) and register transfer level (RTL) design methods. This integration strategy was strategically chosen to take advantage of the efficiency of HLS and the fine control provided by RTL. Our design process follows the guidelines in the Lab1 manual\cite{ECE4810J_FA2024_Lab1} to build a Wishbone IP block.

\subsection{Overall Design Approach}

In this project, we implemented the Wishbone bus architecture using High-Level Synthesis (HLS) to facilitate the design and verification process. HLS allows us to describe the hardware behavior using high-level programming languages such as C and C++, which are then synthesized into Register Transfer Level (RTL) code. This approach significantly reduces the development time and complexity compared to traditional RTL design. The initial phase of our design utilized HLS rapid prototyping and iterating on high-level abstractions of the Wishbone bus architecture. This allows us to efficiently explore multiple design options and build behavior-level models. After that, we moved into the RTL design phase, working out the hardware details. This included implementing precise control logic and timing constraints that are critical to the proper operation of the Wishbone bus. Our approach ensures a harmonious blend of design efficiency and hardware precision, closely following the Lab1 workflow, which provides the basic framework for our development process. By starting with high-level abstractions and then refining them into detailed RTL, we were able to achieve a more efficient and accurate design process.


\subsection{Interface Design}
The Wishbone bus is a flexible and efficient on-chip interconnect (SoC) protocol, achieving communication between master and slave devices through a series of meticulously designed interfaces as shown in Figure \ref{interface}. Here is a detailed description of the main interfaces of the Wishbone bus and their purposes:

\begin{figure}[h]
    \centering
    \includegraphics[width=0.4\textwidth]{cover.png}
    \caption{Interface Details of Wishbone}
    \label{interface}
\end{figure}

\begin{itemize}
    \item \textbf{CLK\_I} (Clock Signal): Provides a synchronous clock signal for all devices on the Wishbone bus, ensuring the synchronization of data transmission.
    
    \item \textbf{RST\_I} (Reset Signal): Used to asynchronously reset all devices on the Wishbone bus, typically triggered by an external control signal.
    
    \item \textbf{ADR\_O/ADR\_I} (Address Lines): Used to transmit address information from the master to the slave device, with ADR\_O indicating the address signal output by the master and ADR\_I representing the address signal input to the slave.
    
    \item \textbf{DAT\_O/DAT\_I} (Data Lines): For transmitting data between the master and slave devices, where DAT\_O signifies the data output by the master, and DAT\_I represents the data input from the slave.
    
    \item \textbf{WE\_O/WE\_I} (Write Enable Signals): Control the direction of data transfer, with WE\_O output by the master to indicate a write operation (1 for write, 0 for read), and WE\_I is an input to the slave, indicating whether the slave accepts the write operation.
    
    \item \textbf{SEL\_O/SEL\_I} (Data Line Select Signals): Used for byte selection, selecting which part of the data lines is active on a byte-by-byte basis. SEL\_O is output by the master, and SEL\_I is an input to the slave.
    
    \item \textbf{STB\_O/STB\_I} (Strobe Signals): Indicate the beginning of a bus operation. STB\_O is output by the master, and STB\_I is an input to the slave.
    
    \item \textbf{CYC\_O/CYC\_I} (Cycle Signals): Indicate the start and end of a bus cycle. CYC\_O is output by the master, signaling the start of a bus cycle, and CYC\_I is an input to the slave, signaling the end of the cycle.
    
    \item \textbf{ACK\_O/ACK\_I} (Acknowledge Signals): Used to indicate the successful completion of a bus operation. ACK\_O is output by the slave, and ACK\_I is an input to the master.
    
    \item \textbf{TAGN\_O/TAGN\_I} (Tag Signals): For conveying user-defined information, which can be utilized to implement more complex communication protocols.
\end{itemize}

Additionally, the Wishbone bus supports various connection methods, including point-to-point, data flow, shared bus, and crossbar connections, to accommodate different system architecture needs. For instance, point-to-point connections are suitable for single master-slave communications, while shared bus and crossbar connections allow multiple masters to communicate with one or multiple slaves.

\subsection{Data Transmission}
\subsubsection{Read Operation}
The read operation is initiated by the master device. It sets the address signal ADR\_O (address output), the data select signal SEL\_O (byte select), and the strobe signal STB\_O to commence a read operation, while grounding the write enable signal WE\_O to explicitly indicate that this is a read operation. At the rising edge of the clock cycle 0, the master asserts the strobe signal STB\_O and the cycle signal CYC\_O to initiate a transaction, marking the beginning of the operation.

Before the subsequent clock cycle's rising edge arrives, the slave device detects the master's initiation signal and responds by driving the requested data onto the data line DAT\_O and asserting the acknowledge signal ACK\_O in response to the master's request. At the rising edge of the clock cycle 1, once the master detects the ACK\_O signal to be high, it samples the data on the data line DAT\_I, and then concludes the operation by de-asserting CYC\_O and STB\_O, signifying the end of a complete read cycle.

This process ensures that the master device can read data from the slave device. After the slave confirms that the data is ready, it informs the master that it can proceed with data sampling via the ACK\_O signal. Once the master device has received and sampled the data, it halts the current transaction, preparing for the next operation or waiting for new instructions. This coordinated communication mechanism is at the core of the Wishbone bus protocol, allowing for efficient data exchange between different components in complex SoC designs.

\subsubsection{Write Operation}
The write operation is meticulously orchestrated by the master device. At the rising edge of the clock cycle 0, the master initiates the transaction by setting the address signal ADR\_O (address output), data signal DAT\_O (data output), data select signal SEL\_O (byte select), strobe signal STB\_O, and the write enable signal WE\_O, which is asserted to indicate a write operation. This configuration signals the commencement of the transaction.

Upon recognizing the master's command, the slave device promptly latches the data from DAT\_I with the provided data and prepares to acknowledge the transaction. Prior to the subsequent rising edge of the clock, the slave device, having recognized the master's command, locks in the data onto the DAT\_O line and asserts the acknowledge signal ACK\_O to confirm receipt of the data and the successful initiation of the write operation.

At the rising edge of the clock cycle 1, the master, upon detecting the assertion of ACK\_O, proceeds to sample the acknowledgment signal ACK\_I. With the confirmation of the successful write operation, the master then deactivates the cycle signal CYC\_O and the strobe signal STB\_O, effectively finalizing the write operation. This coordinated exchange ensures that the data is accurately transmitted from the master to the slave, and the slave's acknowledgment signifies the completion of the data transfer.

This methodical and precise sequence of signals and responses guarantees the integrity and reliability of data transmission within the Wishbone bus system. The write operation is a fundamental aspect of the communication protocol, enabling the master device to update or store data in the slave device's memory or registers.


\subsection{High-Level Synthesis (HLS) Implementation}
The core functionality of the Wishbone bus was implemented using C++ in the HLS environment. The main components included:
\begin{itemize}
    \item \textbf{WishboneMaster}: Handles the generation of bus cycles and data transfer requests.
    \item \textbf{WishboneSlave}: Responds to the master's requests and performs read/write operations.
    \item \textbf{WishboneArbiter}: Manages the arbitration between multiple slaves and ensures correct data routing.
\end{itemize}


We divided our implementation into three main versions, each building upon the previous one to add more features and improve performance:

\begin{itemize}
    \item \textbf{Version 1 (v1)}: Primitive Read/Write Wishbone Master/Slave Implementation.
    \item \textbf{Version 2 (v2)}: Added support for \texttt{BASEADDR}, \texttt{SIZE}, and \texttt{OFFSET}.
    \item \textbf{Version 3 (v3)}: Introduced the \texttt{sel} signal for partial byte operations.
\end{itemize}


Each version was thoroughly tested using corresponding testbenches to ensure correctness and performance. The HLS tool used for this project was Vitis HLS, which provided an efficient workflow for synthesizing and verifying our designs.


\begin{figure}[htbp]
    \centering
    \includegraphics[width=0.4\textwidth]{v3_vitis_hls_co-simulation_report.png}
    \caption{Vitis HLS Co-simulation Report for v3}
    \label{fig:v3_vitis_hls_co-simulation_report}
\end{figure}

\begin{figure}[htbp]
    \centering
    \includegraphics[width=0.4\textwidth]{v3_block_design.png}
    \caption{Block Design for v3}
    \label{fig:v3_block_design}
\end{figure}

\begin{figure}[htbp]
    \centering
    \includegraphics[width=0.4\textwidth]{v3_implemented_design_highlight.png}
    \caption{Implemented Design Highlight for v3}
    \label{fig:v3_implemented_design_highlight}
\end{figure}

\subsection{RTL Implementation}
In the RTL implementation section of our paper, we provide an overview of the hardware description language implementation of the Wishbone interface, designed to facilitate efficient communication between master and slave devices within a System-on-Chip (SoC) environment. The RTL module, `wishbone\_interface`, implements the Wishbone protocol through a series of ports that include clock, reset, cycle, strobe, write enable, address, data input, and output signals, as well as acknowledge and error signals.

The write operation is initiated by the master device at the rising edge of the clock by setting the address, data, and control signals. Upon detecting the command, the slave device responds by latching the data and asserting an acknowledge. The read operation follows a similar pattern, where the master device sets the address and control signals, and the slave provides the data and an acknowledgment. The error signal is utilized to indicate any anomalies occurring during the data transfer process. The corresponding code is shown in Figure \ref{read_write_code}.

\begin{figure}[htbp]
    \centering
    \includegraphics[width=0.4\textwidth]{read_write_code.png}
    \caption{RTL Code of Read and Write}
    \label{read_write_code}
\end{figure}



\section{Methodology}

The methodology for implementing the Wishbone bus architecture using HLS involved several key steps:


\subsection{Testing and Verification}
Each version of the implementation was tested using corresponding testbenches written in C. These testbenches simulated various scenarios, including normal operations, boundary conditions, and invalid addresses, to verify the correctness and robustness of the design.

\subsection{Performance Analysis and Optimization}
Performance metrics such as speed, latency, and throughput were measured to evaluate the efficiency of the Wishbone bus implementation. Optimization strategies were proposed and implemented to address any identified bottlenecks, ensuring that the system met the desired performance standards.




\section{Evaluation}

The evaluation of our Wishbone bus implementation involved several key aspects:

\subsection{Functional Verification}
Functional verification was performed using the testbenches developed for each version of the implementation. The testbenches covered a wide range of scenarios, including:
\begin{itemize}
    \item Basic read/write operations.
    \item Partial byte operations using the \texttt{sel} signal.
    \item Boundary condition testing.
    \item Cross-slave boundary operations.
    \item Handling of invalid addresses.
\end{itemize}

\begin{figure}[h]
    \centering
    \includegraphics[width=0.4\textwidth]{v3_vitis_application_outcome.png}
    \caption{Vitis HLS Application Outcome for v3}
    \label{fig:v3_vitis_application_outcome}
\end{figure}

\subsection{Performance Analysis}
Performance metrics such as speed, latency, and throughput were measured to evaluate the efficiency of the Wishbone bus implementation. Each version of the implementation was synthesized using Vitis HLS, and the Timing Estimate from the Synthesis Summary was analyzed.

\subsubsection{Timing Estimate}
For versions v1 and v2, the Target Timing Estimate was 12ns. However, with the introduction of the \texttt{sel} signal in version v3, the Target Timing Estimate increased to 14ns. This increase in timing can be attributed to the additional logic required to handle partial byte operations, which introduces more complexity and potential delays in the data path.

\begin{figure}[h]
    \centering
    \includegraphics[width=0.4\textwidth]{v3_vitis_hls_solution_report.png}
    \caption{Vitis HLS Solution Report for v3}
    \label{fig:v3_vitis_hls_solution_report}
\end{figure}

\subsubsection{Trade-off Analysis}
The introduction of the \texttt{sel} signal provides significant benefits in terms of flexibility and functionality, allowing for partial byte operations. However, this comes at the cost of increased timing, as seen in the Timing Estimate for v3. The trade-off between timing and functionality must be carefully considered based on the specific requirements of the application.

\paragraph{When to Use \texttt{sel} Signal}
The \texttt{sel} signal is particularly useful in scenarios where:
\begin{itemize}
    \item \textbf{Partial Byte Operations}: Applications that require frequent partial byte read/write operations can benefit from the \texttt{sel} signal, as it allows for more efficient data manipulation without the need for additional software handling.
    \item \textbf{Memory-Mapped I/O}: In systems with memory-mapped I/O, the \texttt{sel} signal can be used to access specific bytes within a word, improving the granularity of control and data access.
    \item \textbf{Performance Optimization}: For applications where the overhead of additional timing is acceptable, the \texttt{sel} signal can provide enhanced functionality and flexibility, leading to overall system performance improvements.
\end{itemize}

In conclusion, while the introduction of the \texttt{sel} signal in version v3 results in a slight increase in timing, the benefits it provides in terms of flexibility and functionality can outweigh the trade-offs in many applications. The decision to use the \texttt{sel} signal should be based on a careful analysis of the specific requirements and constraints of the target application.






\section{Statement of Work}
For the successful completion of the Wishbone Bus Implementation project, our team of three members has divided the work into three parts: System Design and Integration, Hardware Implementation and Debugging, Performance Analysis and Optimization.

\begin{itemize}
    \item Zhangsiyuan Jin: System Design and Integration

    She conducted a preliminary research on the Wishbone bus architecture through data collection and literature search, focusing on its protocols and interfaces. Then she designed a Wishbone compliant system architecture and integrated the bus with other SoC components. In addition, she developed a testbench and conducted simulations to verify the performance of the bus.

    \item Lanqi Ding: Hardware Implementation and Debugging

    She was responsible for crafting the hardware description language (HDL) code for the Wishbone bus interface, bringing the design to life. She expertly implemented the arbitration logic and data transfer mechanisms that are essential for the bus's operation, ensuring efficient communication between components. When challenges arose during simulations and testing, Lanqi diligently debugged the design, pinpointing and resolving issues to maintain project momentum. 

    \item Muzi Li: Performance Analysis and Optimization

    She played a critical role in assessing the performance of the Wishbone bus implementation, diligently measuring key metrics like speed, latency, and throughput to ensure the system's efficiency. She identified performance bottlenecks and proposed targeted optimization strategies to boost the bus's capabilities. Working closely with Member A and Member B, they collaboratively implemented these enhancements, leading to significant improvements in the system's overall performance. Additionally, Member C was tasked with compiling the final report, which comprehensively detailed the performance analysis findings and the tangible benefits achieved through the implemented optimizations.

\end{itemize}


\section{Conclusion}










\printbibliography[title={References},resetnumbers=true,heading=subbibliography]
\clearpage
\end{document}
